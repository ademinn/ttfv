\centerline{\bf Серия 1. Первое знакомство.}

\q1. Имеются чашечные весы без гирь и a)~3 одинаковые по внешнему
виду монеты;  b)~8 одинаковых по внешнему;  c)~9 одинаковых по внешнему
виду монет, одна из которых фальшивая: она легче настоящих
(настоящие монеты одного веса). Сколько надо взвешиваний, 
чтобы наверняка определить фальшивую монету? 

\q2. В озере растут лотосы. За сутки каждый лотос делится пополам, 
и вместо одного лотоса появляются два точно таких же. Ещё через сутки каждый
из получившихся лотосов делится пополам и так далее. Через 30 суток
озеро полностью покрылось лотосами. Через какое время озеро было
заполнено наполовину?

\q3. Баба Яга в своей избушке на курьих ножках завела сказочных
животных. Все они, кроме двух,~--- Говорящие Коты; все, кроме двух,~---
Мудрые Совы; остальные~--- Усатые Тараканы. Сколько обитателей
в избушке у Бабы Яги?

\q4.  Из книги выпала часть. Первая из выпавших страниц имеет
номер 387, а номер последней состоит из тех же цифр, но записанных
в другом порядке. Сколько листов выпало из книги?

\q5. В турнире участвовали 77 шахматистов. Каждые два участника 
турнира сыграли между собой по одной партии.  Сколько
очков набрали шахматисты все вместе?

\q6. В клетках квадрата $3 \times 3$ расставлены числа так, что в любой
строчке и столбце сумма равна 6, а в любом квадрате $2 \times 2$ --- 9.
Какие числа стоят в квадрате?

\centerline{\bf Серия 2, в которой появляется первая важная идея.}

\q1. Отличник Данила заполнил клетки таблицы цифрами так, что
сумма цифр, стоящих в любых трех соседних клетках, равнялась 15,
а двоечник Гаврила стёр почти все цифры. Сможете ли вы восстановить
таблицу?
$$
\begin{tabular}{|c|c|c|c|c|c|c|c|c|c|c|c|c|c|c|}%{15}
\hline
6& & & & & & & &4& & & & & & \\
\hline
\end{tabular}
$$

\q2. Дядька Черномор написал на листке бумаги число 20. Тридцать
три богатыря передают листок друг другу, и каждый или прибавляет
к числу или отнимает от него единицу. Может ли в результате 
получиться число 10?

\q3. Лиса Алиса и Кот Базилио~--- фальшивомонетчики. Базилио
делает монеты тяжелее настоящих, а Алиса~--- легче. У Буратино есть
15 одинаковых по внешнему виду монет, но какая-то одна~--- фальшивая. 
Как двумя взвешиваниями на чашечных весах без гирь Буратино
может определить, кто сделал фальшивую монету~--- Кот Базилио или
Лиса Алиса?

\q4.  Можно ли выложить в ряд все 28 косточек домино согласно
правилам игры так, чтобы на одном конце ряда оказалось 5, а на другом
6 очков?

\q5. Как круглый разделить блинчик тремя прямолинейными разрезами на a)~4; 
b)~5; c)~6; d)~7 частей?


\centerline{\bf Серия 3, в которой появляются Малыш и Карлсон.}

\q1. Голодные Малыш и Карлсон съели торт и стали сытыми. Известно,
что голодный Карлсон легче сытого Малыша, а сытый Карлсон весит
столько же, сколько два голодных Малыша. Что весит больше: торт или
голодный Малыш? Не забудьте обосновать свой ответ.

\q2. Данила посчитал сумму $2+4+6+\ldots+96+98+100$
(это все четные числа из первой сотни),
а Гаврила посчитал сумму $1+3+5+\ldots+97+99$ 
(это все нечетные числа из первой сотни).
Какая из сумм больше и на сколько?

\q3. На какую цифру оканчивается число a) $9^{10}$~?
b) $6^{2011}$~? c)~$7^{239}$~?


\q4. На Марсе 50 государств. Каждый год какие-нибудь три государства
объединяются в одно. Сколько государств будет через 10 лет?

\q5. Даны 13 прямоугольников: $1\times 1$, $1\times 2$, $1\times 3$, \ldots, $1\times 13$.
Можно ли из них всех сложить прямоугольник, все стороны которого больше~1?

\centerline{\bf Серия 4. Cчётно-конструктивная.}

\q1.  В школе, где учится больше 225, но меньше 245 учеников, часть
учеников являются отличниками, а~остальные~--- хорошистами. После сложной
контрольной работы 2/7 отличников стали хорошистами,
а~хорошисты так и остались хорошистами за исключением одного человека,
который стал троечником. При этом хорошистов и отличников
стало поровну. Сколько учеников могло быть в~школе?
Приведите все возможные варианты ответа.

\q2. В детский сад на праздник завезли карточки для обучения чтению: на
некоторых написано "<МА">, на остальных~--- "<НЯ">. Каждый ребёнок взял
три карточки и стал составлять из них слова. Оказалось, что слово
"<МАМА"> могут сложить из своих карточек 20 детей, слово "<НЯНЯ">~---
30 детей, а слово "<МАНЯ">~--- 40 детей. У скольких ребят все три
карточки одинаковы?

\q3.  Лесные звери собрали белоснежный кубик $3\times 3\times 3$ из 27 кубиков $1\times 1\times 1$, 
а Баба Яга уронила его в ведро с синей краской.
Сколько маленьких кубиков имеют ровно a)~0; b)~1; c)~2; d)~3 синих граней?

\q4. Тридцать человек сидят за круглым столом. По очереди, каждый из них говорит: "<Среди моих соседей слева и справа есть ровно один лжец">. 
Известно, что лжецы всегда лгут, а остальные всегда говорят правду. Кроме того, все знают, являются ли лжецами их соседи. 
Сколько лжецов за столом? 

\q5. Данила со своим другом Гаврилой ехали на кружок на трамвае, и купили у
кондуктора два последовательных билетика (номер каждого билетика ---
шестизначное число). Гаврила заметил, что сумма цифр его билета делится на 7.
Данила посмотрел на свой билетик и закричал, что и у него сумма цифр тоже
делится на 7. Могло ли такое случиться~?

\centerline{\bf Серия 5. в которой возвращается и получает дальнейшее развитие 
важная идея.}

\q1. Данила купил в магазине несколько бутылок лимонада по
12 р.~33 коп., несколько шоколадок по 1р.~20 коп. и несколько конфет по
63 коп.  В кассе ему сказали, что он должен уплатить ровно
100 р. Докажите, что его обсчитали.

\q2. Из набора гирек с массами $1, 2, \ldots , 77$ г потерялась гирька
массой 30 г. Можно ли оставшиеся 76 гирек разложить на две кучки
по 38 гирек в каждой так, чтобы массы обеих кучек были одинаковы?

\q3. Можно ли расставить на шахматной доске 15 коней так, чтобы каждый
конь бил ровно одного из оставшихся?

\q4.  Какое наименьшее число разрезов должен сделать Гаврила, 
чтобы разрезать квадрат $4\times 4$ на квадраты $1\times 1$, 
если Данила запретил ему за один раз разрезать несколько приложенных 
друг к другу частей? 

\q5. В комнате стоят трехногие табуретки и четырехногие стулья всего 22 
предмета мебели.
Когда на все эти сидячие места уселись люди, в комнате оказалось 120
ног. Сколько в комнате табуреток?

\q6. Разрежьте "<по клеточкам"> квадрат $5\times 5$ на 7 различных прямоугольников.

\centerline{\bf Серия 6, в который мы наконец-то поиграем.}

\q1. Малыш и Карлсон положили на стол a)~15 конфет; b)~239 и играют в такую игру:
они ходят по очереди, за ход можно съесть одну или две конфеты.
Выигрывает тот, кто съел последнюю конфету.
Первым ходит Малыш.
Кто из игроков может обеспечить себе победу, и как ему для этого
надо играть?

\q2. На прямой расположено пять точек $A, B, C, D, E$ (именно в таком порядке).
Известно, что $AB=19$ см., $CE=97$ см., $AC=BD$. Найдите длину отрезка $DE$.

\q3. Гаврила стер одну цифру в числе $19345202659\star 4639199$. 
Сможет ли Данила заменить $\star$ на цифру так, чтобы полученное число делилось на 3?

\q4. За большим круглым столом сидят 239 жителей острова рыцарей и лжецов
(рыцари всегда говорят правду, лжецы всегда лгут). Каждого из них спросили,
кто его сосед справа: рыцарь или лжец. Докажите, что хотя бы один из них
ответил, что сосед~--- рыцарь.

\q5. В круге отметили точку. Можно ли так разрезать этот круг
на три части, чтобы из них можно было бы сложить новый круг, 
у которого отмеченная точка стояла бы в центре?

\centerline{\bf Серия 7. Делай как я!}

\q1. За круглым столом сидят 5 мальчиков и 5 девочек. Докажите,
что какой-то мальчик сидит напротив девочки.

\q2. Гаврила теперь решил резать квадрат $4\times 4$ на квадраты $1\times 1$, 
по своим правилам~--- с наложениями. 
Какое наименьшее число разрезов должен теперь сделать
Гаврила, чтобы разрезать квадрат на квадратики $1\times 1$?

\q3.  Данила и Гаврила играют в увлекательную игру: по очереди кладут одинаковые
пятикопеечные монеты на круглый  стол. Монеты не должны  налегать друг на
друга или вылезать за край стола.  Проигрывает тот, кому некуда класть
монету. Первым ходит Данила. Как он  должен играть, чтобы заведомо выиграть?

\q4.  Собрались несколько рыцарей и лжецов, все разного роста.
Каждый заявил:  "<Среди нас есть лжец, который выше меня">.
Сколько среди них лжецов?

\q5. Гаврила и Данила пошли в тир. Договорились, 
что Гаврила делает 5 выстрелов и за каждое попадание в цель 
получает право сделать ещё 2 выстрела. 
Всего Гаврила сделал 17 выстрелов. Сколько раз он попал
в цель?

\q6. Данила записал трехзначное 
натуральное число и вычел из него его собственную сумму цифр.
Докажите, что полученная Данилой разность делится на 9.

\centerline{\bf Серия 8, имени одного известного математика. }

\small
\textit{Обратите внимание, что эта серия на среду 7 декабря. 
В субботу 3 декабря состоится тренировочная олимпиада.
Началов в 16:00, продолжительность 3 часа.}
\normalsize

\q1.  В классе, в котором учится кузина Мила, всего 30 человек. В контрольной
работе по математике Мила, пользуясь своим калькулятором, сделала 13 ошибок, остальные сделали меньше.
Докажите, что в этом классе есть трое человек, которые сделали ошибок поровну.

\q2. Данила и Гаврила выложили в ряд 25 пирожных "<картошка">, каждое на свою тарелку, и стали 
играть в такую игру. Они ходят по очереди, и каждый своим ходом может 
съесть любое пирожное, либо любые два пирожных, лежащих на соседних тарелках. 
Тот, кому после очередного хода противника ничего 
не останется, проигрывает.
Первым ходит Данила. Как он должен играть, чтобы заведомо выиграть?

\q3.  Гаврила положил на счет в банке 3 рубля. Каждый день на его
счет добавляется количество рублей, равное сумме цифр текущего счета.
Когда Гаврила пришел забирать вклад, ему выдали ровно 1000 рублей. Докажите, что
его обсчитали.

% \q4.  Среди следующих утверждений ровно одно неверно. Найдите его: \par
% - если точный квадрат делится на 6, то он делится на 36. \par
% - если точный квадрат делится на 7, то он делится на 49. \par
% - если точный квадрат делится на 8, то он делится на 64. \par

\q4. Умная девочка Маша записала \textbf{произвольное} 
натуральное число и вычла из него число, образованное его 
двумя последними цифрами. (Например из числа 1239 она вычла число 39.)
Докажите, что полученная Машей разность делится на 4.

\q5. В записи $2 : 3 : 4 : 5 : 6 = 5$
расставьте скобки так, чтобы получилось верное равенство.

\centerline{\bf Серия 10. Шахматная.}

\q1. Гаврила вырезал из шахматной доски \quad
a)~одну угловую клетку; \quad
b)~две угловые клетки: левую верхнюю и правую верхнюю. \quad

Можно ли оставшуюся часть доски разрезать на
прямоугольники размером $1\times 2$ клетки?

\q2. Какое наибольшее количество не бьющих друг друга ладей можно поставить
на шахматной доске? (Размер шахматной доски~--- $8\times 8$ клеток,
ладьи бьют друг друга, если они стоят на одной горизонтали
или на одной вертикали.)

\q3. Маша и Медведь раскрашивают лампочки в новогодней гирлянде. 
Гирлянда состоит из 80 лампочек. За один раз можно закрасить 3 или 4 лампочки,
расположенные рядом. Тот, кто закрасит последнюю лампочку - забирает себе всю гирлянду. 
Начинает красить Медведь. На чьей ёлке будет красоваться гирлянда в новогоднюю ночь? 

\q4.  Докажите, что число $\overline{abcabc}$ делится на 77.

\q5.  Докажите, что любое число можно представить в виде отношения точного квадрата и точного куба.                          

\centerline{\bf Серия 11. Ещё более шахматная.}

\q1. Какое наибольшее количество не бьющих друг друга королей
можно поставить на шахматной доске $8\times 8$?

\q2.  Гаврила вырезал угловую клетку из квадрата размера $5\times 5$ клеток. Сможет ли Данила
разрезать оставшуюся часть на уголки из трех клеток?

\q3. А Данила вырезал из шахматной доски 
две угловые клетки: левую-верхнюю и правую-нижнюю. Сможет  ли Гаврила оставшуюся часть доски разрезать на
прямоугольники размера $1\times 2$ клетки?

\q4.  В поле стояли 20 столбов, каждые два столба были соединены
одним проводом. Часть проводов украли, и теперь каждый столб
соединен только с пятью другими. Сколько проводов осталось?

\q5. Докажите что число $\overline{ababab}$ делится на 21.

\q6. Девочки и мальчики вышли парами из леса, где они собирали грибы. 
Причем в каждой паре либо у мальчика было либо в два раза больше грибов,
чем у девочки, либо в два раза меньше. Гаврила шел мимо и насчитал в корзинках у ребят 56 грибов.
Докажите, что он ошибся


\centerline{\bf Серия 12.  Детки в клетку.}

\q1.  Лягушонок поселился на клетчатом болоте, в вершинах которого расположены кочки.
Он умеет перепрыгивать с любой кочки на любую из соседних кочек (т.е.\ по сторонам клеточек).
Сможет ли он вернуться на исходную кочку ровно через 239 прыжков?

\q2. Докажите, что никакой прямоугольник с целыми сторонами нельзя разрезать
на фигурки вида \lower0.1\cellsize\cells{
 __
|_|_|_
   |_|_|
}.


\q3. Можно ли прямоугольник размера $5\times 9$ клеток
разрезать на трехклеточные уголки вида
\lower0.1\cellsize\cells{
 _
|_|_
|_|_|
}~?

\noindent\parbox{155truemm}{\q4. В новогодней елке Дед Мороз и Кащей Бессмертный по очереди заменяют
звездочки на цифры. Первым ходит Дед Мороз. Дед Мороз выиграет,
если все равенства будут верными, а Кащей Бессмертный --- если хотя бы одно будет неверно. Если Дед Мороз проиграет, то елка не загорится, Новый
Год не наступит. Будет ли встреча Нового Года?

\q5.  Какое наибольшее количество не бьющих друг друга коней
можно поставить на шахматной доске?
}
\parbox{4cm}{\begin{align*}
&\ \star \\
                      \star&=\star    \\
                     \star\ \star&=\star\ \star   \\
                   \star\ \star\ \star&=\star\ \star\ \star  \\
                  \star\ \star\ \star\ \star&=\star\ \star\ \star\ \star       
     \end{align*}
}






\centerline{\bf Серия 13, в которой мальчики и девочки следят за тем, куда перемещаются булочки.}

\q1. Лягушонок подрос и теперь умеет прыгать только по диагоналям
клеточек. Сможет ли он теперь вернуться на свою кочку через 239 прыжков?

\q2. С числами можно выполнять следующие операции:
умножать на два или произвольным образом переставлять цифры
(нельзя только ставить ноль на первое место).
Можно ли такими операциями из числа 1 получить 411?

\q3.  За столом сидели 5 мальчиков и 6 девочек, а на столе на тарелке
лежало несколько булочек.
Каждая из девочек дала по булочке (с тарелки) каждому знакомому мальчику.
Затем каждый мальчик дал по булочке (с тарелки) каждой незнакомой ему девочке.
После этого тарелка опустела. Сколько было булочек?

\q4. Какое наибольшее количество не бьющих друг друга слонов
можно поставить на шахматной доске?

\q5. Докажите, что число a) $4^{2011}+6^{2011}$ b) $9^{2011}+1$ 
c) $3^{2009}+7^{2009}$ делится на 10.

\q6.  В школе 30 классов и 1000 учеников. Докажите, что есть класс, в котором есть как минимум 34 ученика.

\q7. На слете инопланетян встретились 2011 марсианин. У каждого марсианина
имеется 3 руконожки. Могут ли они взяться за руконожки так, чтобы не осталось
ни одной свободной руконожки (в каждом руконожкопожатии участвуют две
руконожки)?


\centerline{\bf Зимняя серия математических боев. Новогодний матбой.}


\q1. Два ёлочных базара охраняют бригады сторожей. Каждый
базар --- одинаковое число бригад. В каждой бригаде одинаковое число
сторожей. Все сторожа проспали ночей больше чем количество бригад,
охраняющих один елочный базар, но меньше, чем сторожей в бригаде.
Всего все вместе проспали 2002 человеко-ночи. Сколько сторожей в бригаде?

\q2.  В ряд стоят 8 мешков с подарками, в двух соседних мешках количество
подарков отличается на один. Может ли там всего быть 225 подарков?

\q3. Докажите, что Чёрный Дракон не может, начав с левого 
нижнего угла, ходом шахматного коня обойти квадрат $2011 \times 2011$ 
и вернуться на исходную клетку. 

\q4. В лесу 21 поляна, некоторые из которых соединены тропинками.
С каждой поляны выходит 10  тропинок. Докажите, что Дед Мороз может
с любой поляны по тропинкам попасть на любую (возможно, по пути заходя
на другие поляны).

\q5. Чёрный Дракон принёс лесным жителям 24 килограмма счастья. 
При этом ровно 9 килограммов счастья причитается семье серых зайцев. 
Сможет ли Дракон отмерить положенное зайцам счастье с помощью чашечных весов без гирь? 

\q6.  В деревеньке Данилово живет в два раза больше жителей, чем в деревушке Гаврилово. 
Эльфы Деда Мороза подсчитали, что всего в этих деревнях Дед Мороз подарит 2012 подарков. 
Вопрос: не помогали ли эльфам при подсчёте олени? 

\q7.  Дед Мороз и Санта-Клаус  по очереди ставят королей на доску $9 \times 9$ так, 
чтобы они не били друг друга. Проигрывает тот, кто не может сделать хода. 
Кто из игроков имеет выигрышную стратегию, если начинает ходить обладатель российской прописки?

\q8.  Снеговик задумал натуральное число, умножил его на 13, зачеркнул 
последнюю цифру, полученное число умножил на 8, опять зачеркнул последнюю 
цифру и получил 20. Какое число задумал Снеговик? 

\centerline{\bf Зимняя серия математических боев. Рождественский матбой.}

\q1. На какую цифру оканчивается число $1^2+2^2+3^2+ \cdots
+{99}^2+{100}^2$?

\q2. За круглым столом сидят 100 человек --- 51 мальчик и 49 девочек.
Докажите, что найдутся два мальчика, сидящие напротив друг друга.

\q3. В полоске $1 \times 20$ стоит 20 фишек. За ход можно поменять две
фишки стоящие через одну клетку местами. Можно ли все фишки поставить в
обратном порядке?

\q4. 10 невоспитанных верблюдов плюются друг в друга. Каждый плюнул ровно
в пятерых. Докажите, что есть верблюд, в которого плюнуло хотя бы пятеро.

\q5. Каких чисел больше среди натуральных чисел от 1 до 2002: делящихся
на 9, но не делящихся на 8 или делящихся на 8, но не делящихся на 9?

\q6. В некотором месяце три воскресения пришлись на четные числа. Каким
днем недели было 17-ое число этого месяца?

\q7. Можно ли в квадрате $5 \times 5$ расставить числа от 1 до 25 так,
чтобы в любой строчке сумма каких-то двух чисел была равна сумме
остальных трех?

\q8. Три кружковца: Рита, Федя и Гоша спорили после занятия кто решил
больше задач: \par
Рита: Я решила больше всех. \par
Федя (Рите): Нет, не ты. \par
Гоша: На самом деле я. \par
Рита: Ну, уж точно не Гоша. \par
Федя: Потому что на самом деле я. \par
Известно, что тот, кто решил больше всех, сказал правду,
а остальные --- солгали. Кто решил больше всех задач?
\centerline{\bf Серия 14, в которой мы просыпаемся после каникул.}

{\it \noindent Упражнениями будут называться задачи, которые 
обязательно надо решить.  
Если не сдано упражнение, другие задачи не принимаются. 
В крайнем случае, 
если у Вас возникают трудности в решении упражнения~--- обращайтесь к преподавателям~--- поможем.}
\smallskip

\q0. (Упражнение) Избавьтесь от скобок в выражении:
a) $2(5x+3)$;\quad  b) $(7y-4)\cdot 6$.

\q1. Докажите, что квадрат  $10\times 10$ нельзя разрезать на
фигурки вида a) \lower.1\cellsize\cells{
   _
 _|_|_ 
|_|_|_|
}. b) \lower.1\cellsize\cells{
     _
 _ _|_| 
|_|_|_|
}. 

\q2.   "<Хромой король"> ---  это фигура, которая ходит на одну клетку либо  
вправо, либо  вверх.
Гаврила поставил хромого короля на клетку $a1$ шахматной доски. 
После этого Данила и Гаврила стали по очереди передвигать хромого короля, начинает Данила.  
Проигрывает тот, кто не может сделать ход. Кто из игроков 
имеет выигрышную стратегию?

\q3. Некоторые жители острова заявили, что на острове чётное число рыцарей, 
а остальные заявили, что на острове нечётное число лжецов. Может ли число жителей острова быть нечётным?  

\q4. После нескольких занятий кружка художественного свиста оказалось, что
у каждого в кружке есть ровно один друг и ровно один враг. Докажите, что
в кружке занимается четное число людей.


\centerline{\bf Серия 15, в которой мы ведём строгий учёт знакомых.}
 
\q0. ({\bf Упражнение}). Упростите выражение
a)~$x(4y+1)$; b)~$6(x+5)+3(6x-10)$; c)~$2x(4y+5z)+3x(y+2z)$;
d)~$(3x+5)-(2x+1)$. 

\q1. Докажите, что квадрат a) $6\times 6$; b) $10\times 10$ нельзя разрезать на
фигурки вида \lower.1\cellsize\cells{
 _ _ _ _
|_|_|_|_|
}.

\q2. Умная девочка Маша записала произвольное
натуральное число и вычла из него a) число, образованное его
тремя последними цифрами; b,c) его сумму цифр.
Докажите, что полученная Машей разность делится на а) 8; b) 9; c) 3.

\q3. На занятии кружка каждый из учеников подсчитал,
сколько у него знакомых в этом кружке. Докажите, что у каких-то двух
учеников эти числа совпали.  

\q4. В спортивной школе учатся 100 человек. В школе организованы три секции:
футбола, волейбола и шахмат. Каждый ученик занимается хотя бы в одной из секций
(а некоторые~--- сразу в двух или трех). Футболом занимаются 65,
волейболом~--- 50, а шахматами~--- 45 учеников. При этом 10 учеников
занимаются во всех трех секциях. Сколько тех, которые занимаются ровно в двух секциях?

\q5. В классе каждая девочка дружит с 6 мальчиками,
а каждый мальчик~--- с 4 девочками. Мальчиков
в классе 15. Сколько девочек в этом классе?


\centerline{\bf Серия 16, в которой задача №6 сформулирована корректно.}

\q0. ({\bf Упражнение}). Упростите выражение
a)~$3a(4b+5)-2b(7-6a)$;
b)~$(4a+3b)-(a+b)-(2a-5b)+(b+2a)$.  
c) Расскройте скобки $(a+1)(b+1)$.

\q1. Лягушонок ещё подрос и теперь умеет прыгать только по диагоналям
прямоугольников 1 на 3. Сможет ли он теперь вернуться на свою кочку через 239 прыжков?

\q2. "<Хромой ферзь"> ---  это фигура, которая ходит на одну клетку либо  
вправо, либо  вверх, либо и вправо и вверх.
Гаврила поставил хромого ферзя на нижнюю левую клетку доски $5\times 8$. 
После этого Данила и Гаврила стали по очереди передвигать хромого ферзя, начинает Данила.  
Проигрывает тот, кто не может сделать ход. Кто из игроков 
имеет выигрышную стратегию?

\q3.  Докажите и {\bf запомните} следующую формулу: $a^2-b^2=(a+b)(a-b)$. 

\q4. На шахматной доске стоит 51 ладья. Докажите, что каждая из этих ладей
бьет какую-нибудь другую.

\q5. Гаврила не поленился вычислить сумму
$9 + 99 + 999 + \dots + 99...9$
(в последнем слагаемом 99 девяток) и выписать результат на доску.
Сколько раз на доске записана цифра 1?

\q6. Покажите, как раскрасить клетки таблицы $6\times 6$ в три цвета так,
чтобы рядом с любой клеткой находилось ровно по одной клетке каждого
другого цвета. Мы говорим, что клетки расположены рядом, если у них
есть общая сторона.

\centerline{\bf Серия 17, в которой лягушонок наконец-то вернется домой, а в нем уже кто-то живёт.}

\q0. ({\bf Упражнение)} Раскройте скобки в выражении a) $(x+1)(2y-1)$; 
b) $(a-1)(z-3)$; c) $(x+y)(x+z)$; d) $(2a+b)(3a-b)$.

\q1. Кузина Мила записала натуральное число. 
После этого Гаврила посчитал сумму цифр этого числа, стоящих на местах с нечетными номерами (нумерация начинается справа). 
А Данила посчитал сумму цифр числа, которое записала Мила, стоящих на местах с четными номерами. 
Докажите, что если Мила вычтет из своего числа число Гаврилы, а затем прибавит к результату число Данилы, то 
получит число, которое делится на 11. 

\q2. Двое играют в такую игру. Сначала первый называет любое число от 1 до~10.
Потом второй прибавляет к этому числу любое число от 1 до 10 и называет результат.
Потом первый прибавляет любое число от 1 до 10 и называет результат, и так далее.
Выигрывает тот, кто первым назовет трехзначное число.
Кто может обеспечить себе победу: первый или второй?

\q3.  Докажите и {\bf запомните} следующую формулу: $(a+b)^2=a^2+2ab+b^2$.

\q4. В кружке художественного свиста у каждого ровно один друг и ровно один
враг. Докажите, что кружок можно разделить на два одинаковых кружка так, чтобы ни в каком
из двух кружков не было ни друзей, ни врагов.

\q5.  На каждой клетке доски $5\times 5$ сидит один дрессированный лягушонок.
По команде "<Ква"> каждый лягушонок перепрыгивает на одну из соседних клеток,
(клетки считаются соседними, если они имеют общую сторону). Докажите, что
после команды "<Ква"> какие-то два лягушонка окажутся на одной клетке.

\q6. Лягушонок научился прыгать по диагоналям прямоугольников $1\times 4$. 
Докажите, что он может допрыгать в любую наперед заданную кочку.

\centerline{\bf Серия 18. Даже в магазине "<Всё для чая"> есть место математике.}

\q1. a)~Докажите, что среди 3 натуральных чисел обязательно найдутся 2
числа, разность которых делится на 2.
b)~Докажите, что сумма любых трех последовательных натуральных
чисел делится на 3. c)~Верно ли, что сумма любых трех последовательных натуральных
чисел делится на 6?

\q2. В лесу 21 поляна, некоторые из которых соединены тропинками.
С каждой поляны выходит 10  тропинок. Докажите, что Мартовский Заяц может
с любой поляны по тропинкам попасть на любую (возможно, по пути заходя
на другие поляны).

\q3.  Докажите и {\bf запомните} следующую формулу: $(a-b)^2=a^2-2ab+b^2$.

\q4.  a) В магазине "<Всё для чая"> продается 3 разных чашки, 4 разных блюдца и
5 разных чайных ложек. Сколькими способами можно составить чайный набор из
чашки и блюдца?
b) А набор из чашки, блюдца и ложки?

\q5.  В вершинах куба расставлены числа: 7 нулей и единица. За один ход разрешается прибавить 
по единице к числам в концах любого ребра куба. Можно ли добиться того, чтобы все числа  стали равными? 

\centerline{\bf Серия 19. Не так просты простые числа.}

\noindent \textit {Определение}: Простое число~--- это натуральное число, 
большее~1, которое не делится ни на какое натуральное число,
кроме самого себя и единицы. Вот несколько
первых простых чисел: 2, 3, 5, 7, 11, 13, 17, 19, 23, 29, $\ldots$

\smallskip

\q1. Найдите все такие натуральные числа $p$, что 
a)~$p$ и $p+5$~--- простые числа;
b)~$p$, $p+10$ и $p+20$~--- простые числа.

\q2. Докажите, что среди 11 натуральных чисел обязательно найдутся 2
числа, разность которых делится на 10.


\q3.  Данила разрезал клетчатый прямоугольник на фигурки вида 
\lower.5\cellsize\cells{
 _
|_|_ 
|_|_|
}. 
Докажите, что Гаврила сможет разрезать этот же прямоугольник на фигурки вида
\lower.1\cellsize\cells{
 _ _ _ 
|_|_|_|
}.

\q4.  Докажите и {\bf запомните} следующую формулу: $1+2+\ldots+n=\frac{n(n+1)}{2}$. 

\q5.  Есть 5 клумб, в каждую можно посадить красную или белую розу.
Сколько есть разных способов посадить розы
a) если в каждую клумбу надо что-то посадить;
b) если какие-то клумбы можно оставить пустыми?

\q6.  В каждой клетке доски $9\times 9$ сидит дрессированный лягушонок. 
По команде "<Ква-Ква"> каждый лягушонок сделал прыжок на какую-то соседнюю по диагонали клетку. 
Докажите, что после этой команды на доске образовалось хотя бы 9 пустых клеток.